\documentclass{article}

\usepackage[french]{babel}
\usepackage[T1]{fontenc}
\usepackage[sfdefault]{biolinum}
\usepackage{graphicx}
\usepackage{float}
\usepackage{pifont}

\title{Les \textbf{SSII}}
\date{7 mars 2022}
\author{Stéphane MEYER}

\begin{document}
  \pagenumbering{gobble}
  \maketitle
  \newpage
  \pagenumbering{arabic}

\section{Définition}

\paragraph{}

Une société de Service et d'Ingénierie en Informatique ou Entreprise de Service du Numérique (\textbf{ESN} suite à un changement de nom en 2013), désigne une société experte en informatique et en nouvelles technologies. Il s'agit d'une entreprise de service rattachée au secteur tertiaire.

\section{Compétences}

\paragraph{}

Les \textbf{ESN} proposent des expertises sur tous les services du numérique et fournissent leur savoir-faire aux entreprises qui ont besoin, par exemple, de créer un site web, un logiciel ou encore de gérer un parc informatique.

\paragraph{}

Les prestation offertes englobent des métiers très différents :

\begin{itemize}
  \item[\ding{51}] Conseil
    \begin{itemize}
      \item Conseil en organisation
      \item Conseil en processus métier
      \item Conseil en conduite du changement
      \item Conseil technique et R{\&}D externatlisé
    \end{itemize}
  \item[\ding{51}] Intégration de systèmes
    \begin{itemize}
      \item Architecture et urbanisation des systèmes d'informations
      \item Développement d'application/ingénierie logicielle
      \item Mise en place de PGI/ERP \footnote{Progiciel de Gestion Intégré / Enterprise Resource Planning.}
      \item Solution de communication entre systèmes informatiques hétérogènes
      \item Vente de licences de logiciels
      \item Assistance technique
    \end{itemize}
  \item[\ding{51}] Infogérance
    \begin{itemize}
      \item Tierce maintenance applicative (TMA) : maintenance et évolution applicative
      \item Tierce recette applicative (TRA) et testing : gestion externalisée des tests et de la qualité des logiciels
      \item Gestion des infrastructures :
        \begin{itemize}
          \item[\ding{219}] support aux utilisateurs
          \item[\ding{219}] maintenance
          \item[\ding{219}] hébergement
          \item[\ding{219}] gestion des système et réseaux
          \item[\ding{219}] gestion de la sécurité des systèmes
        \end{itemize}
      \item BPO \footnote{Business Process Outsourcing.} : externalisation des processus métier (RH, comptabilité...)
    \end{itemize}
  \item[\ding{51}] Formation
\end{itemize}

\paragraph{}

Les clients des \textbf{ESN} sont des entreprises. Elles peuvent proposer des prestations qui peuvent ne durer que quelques heures pour un problème ponctuel, ou des mois voire des années dans le cadre de l'installation complète et la gestion d'un système.

\paragraph{}

Les \textbf{ESN} sont de tailles très variées, de l'entreprise individuelle au grand groupe.

\section{Composition}

\paragraph{}

Elles sont composées de deux catégories de personnes :

\subparagraph{}
Des consultants qui interviennent dans les entreprises pour répondre à leur problématique en informatique ou pour offrir une assistance technique (ou "régie"). Dans ce domaine l'\textbf{ESN} recrute des personnes disposant de conpétences pou revendre leur travail à la journée selon un taux journalier convenu.
\subparagraph{}
Des commerciaux qui ciblent les entreprises qui ont des besoins et qui négocient les contrats de prestations.

\section{Marché \textbf{ESN}}

\paragraph{}

Le marché \textbf{ESN} est florissant et se développe sans cesse. De nombreux secteurs favorisent la croissance des \textbf{ESN} dont les banques, l'assurance, la finance et le service aux professionnels.

\paragraph{}
Selon l'INSEE \footnote{Institut National de la Statistique et des Etudes Economiques.}, en 2016, le secteur de la programmation, du conseil et des autres activités informatiques sur mesure (hors édition de logiciels) emploie 370 000 salariés et dégage un chiffre d'affaires de 61,7 milliards d'euros.

\paragraph{}
Et selon Syntec \footnote{Syndicat représentant plus de 3000 groupes et sociétés françaises spécialisées dans les professions du numérique.}, en 2018, 78\% des \textbf{ESN} ont accru leur chiffre d'affaires.

\section{Les principales \textbf{ESN}}

\begin{figure}[H]
  \makebox[\textwidth]{\includegraphics[scale=0.25]{img/capgemini.png}}
  \caption{Logo Capgemini}
\end{figure}

\paragraph{Capgemini :}

Fondée en 1967 à Grenoble sous le nom de \textbf{Sogeti}, Capgemini est un cabinet de consil qui a le plus gros chiffre d'affaire de France et elle figure parmi les dix plus grosses entreprises du secteur au niveau mondial.

  \subparagraph{Prestations proposées (liste non exhaustive) :}

  \begin{itemize}
    \item Intelligence artificielle
    \item Innovation appliquée
    \item Opérations commerciales
    \item Services Cloud
    \item Cybersécurité
    \item Industrie intelligente
  \end{itemize}

\begin{figure}[H]
  \makebox[\textwidth]{\includegraphics[scale=0.25]{img/soprasteria.png}}
  \caption{Logo Sopra Steria}
\end{figure}

\paragraph{Sopra Steria :}
L'une des meilleures \textbf{ESN} en France, Sopra Steria dispose d'un effectif de 45 000 employés dont des :
  \begin{itemize}
    \item Développeurs
    \item Data scientists
    \item Data analysts
    \item Business analysts
    \item Ingénieurs systèmes et réseaux
    \item Chefs de projets
    \item Experts en cybersécurité
    \item Experts Cloud
  \end{itemize}

\begin{figure}[H]
  \makebox[\textwidth]{\includegraphics[scale=0.25]{img/ibm.jpeg}}
  \caption{Logo IBM}
\end{figure}

\paragraph{IBM :}

  \subparagraph{IBM propose à ses clients de nombreux services comme :}
    \begin{itemize}
      \item Conception métier
      \item Services Cloud
      \item Espaces de travail numériques
      \item Applications d'entreprise
      \item Service réseaux
      \item Automatisation et conception de processus
      \item Support technique
    \end{itemize}

\begin{figure}[H]
  \makebox[\textwidth]{\includegraphics[scale=0.25]{img/atos.png}}
  \caption{Logo Atos}
\end{figure}

\paragraph{Atos :}
Créée en 1997, son chiffre d'affaires en 2019 s'élevait à près de 11 milliards d'euros.
  \subparagraph{Services :}
  \begin{itemize}
    \item Intelligence artificielle
    \item Informatique avancée
    \item Solutions Cloud
    \item Internet des Objets
    \item Cybersécurité
  \end{itemize}

\begin{figure}[H]
  \makebox[\textwidth]{\includegraphics[scale=0.5]{img/accenture.png}}
  \caption{Logo Accenture}
\end{figure}

\paragraph{Accenture :}
Entreprise mondiale de conseil qui compte plus de 624 000 salariés dans 120 pays. En France, Accenture représente 7000 collaborateurs dans 4 bureaux principaux basés à Paris, Nantes, Sophia Antipolis et Toulouse.

\subparagraph{Services :}
  \begin{itemize}
    \item Conseil en technologies
    \item Automatisation
    \item Cloud
    \item Data {\&} Analytics
    \item Etc.
  \end{itemize}

\section{Conclusion}
\paragraph{}
Nous avons vu ensemble ce qu'était une \textbf{SSII/ESN} et je me ferai un plaisir de répondre à vos éventuelles questions. Merci de votre attention.

\end{document}
